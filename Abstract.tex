\section*{Kurzfassung}
F�r die Gestaltung einer komfortablen, sicheren, energie- und verkehrseffizienten Mobilit�t wird an der Automatisierung des Verkehrs geforscht. Daf�r werden Konzepte erarbeitet, bei denen Fahrzeuge und die Infrastruktur sowohl mit Umfelderfassungssensoren als auch mit Kommunikationsmodulen ausgestattet werden. Dabei entstehen Fragestellungen bez�glich des Aufbaus und der Architektur der entstehenden L�sungen. F�r die Analyse der Systemkonzepte vor ihrem Einsatz, wird im Rahmen dieser Arbeit ein Auslegungstool erzeugt. Hierf�r werden zun�chst die Sensoreigenschaften herausgearbeitet und bestehende Systemkonzepte untersucht. Mit Hilfe dieser Erkenntnisse wurden Anforderungen an ein Auslegungstool ermittelt. Die Anforderungen wurden schlie�lich in Matlab umgesetzt. Mit vier verschiedenen Szenen wurde die Anwendbarkeit evaluiert. Die Evaluation ergab, dass eine Konzeptanalyse bereits m�glich ist, jedoch besteht noch ein gro�es Ausbaupotential, um eine detalliertere Analyse zu erm�glichen.