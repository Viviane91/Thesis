\section{Einleitung}
\label{chap:Einleitung}
Seit �ber 10 Jahren besteht ein gro�es Aktivit�tsfeld im Bereich des vernetzten und automatisierten Fahrens. Das Ziel ist die Automatisierung des Verkehrs, um die Mobilit�t energieeffizient, komfortabel, sicher und verkehrseffizient zu gestalten \cite{Bengler.2018}.

Um dies zu erreichen, werden verschiedene Konzepte erarbeitet, bei denen Fahrzeuge und Infrastruktur mit Sensorik zur Umfelderfassung ausgestattet werden, wie z.B. in \cite{Vivacqua.2017} und \cite{Dotzauer2017}. Diese Konzepte sollen f�r die drahtlose Informationsweitergabe des so erfassten fahrzeugspezifischen Lagebildes zwischen den Verkehrsteilnehmern genutzt werden \cite{Bengler.2018}. Dadurch entstehen generelle Fragestellungen bez�glich des Aufbaus und der Architektur der entstehenden L�sungen. Um diese Systemkonzepte noch vor ihrem Einsatz zu analysieren, wird ein Auslegungstool ben�tigt.

Ziel dieser Arbeit ist die Erstellung eines Simulations-Tools, um Systemkonzepte zielgerichtet auszulegen, zu analysieren und zu bewerten. Hierf�r m�ssen die einzelnen Sensoreigenschaften herausgearbeitet und systematisch abgebildet werden.

Kapitel\,\ref{chap:Grundlagen} erl�utert zun�chst die Funktionsweisen und Eigenschaften der Umfelderfassungssensoren Ultraschall, Radar, Lidar und Kamera. Anschlie�end wird auf die Datenverarbeitung eingegangen. Hierzu geh�rt die Objekterkennung, das Tracking und die Sensordatenfusion. In Kapitel\,\ref{chap:Einsatz} werden bestehende Systemkonzepte bei Fahrzeugen und in der Infrastruktur vorgestellt. Anschlie�end werden in Kapitel\,\ref{chap:Anforderungen} die Anforderungen an ein Simulations-Tool zur Systemkonzeptbewertung herausgearbeitet. Die Umsetzung des Tools wird in Kapitel\,\ref{chap:Software-Tool} vorgestellt. Kapitel\,\ref{chap:Anwendung} umfasst die Anwendung des Tools f�r verschiedene Systemkonzepte. In Kapitel\,\ref{chap:Zusammenfassung} werden m�gliche Erweiterungen des Tools, die im Rahmen dieser Arbeit nicht implementiert worden sind, vorgestellt.

 %mit Hilfe von Daten der Forschungskreuzung in Braunschweig.
