\documentclass[
	pdftex, %pdftex verwenden
	a4paper, %DIN A4
	oneside, %einseitig
	titlepage,
	headsepline, %Linie nach Kopfzeile
	parskip=half, %Absaetze
	captions=oneline, %Ueber- und Unterschriften fuer Tabellen und Abbildungen
	12pt, %groessere Schrift fuer bessere Lesbarkeit am Bildschirm
	bibliography=totoc, %Literaturverzeichnis ohne Nummer im Inhaltsverzeichnis
	toc=listof,
	version=first %neues Kommando von bibtotoc
]{scrartcl}

% Aktuelles Datum ermitteln
\usepackage[ngerman]{datenumber}

% Seitenraender
\usepackage{geometry}
\geometry{includeheadfoot,left=3cm,right=3cm,top=2.5cm,bottom=2.5cm}

% Tabellen
\usepackage{booktabs}
\usepackage{tabularx}
%\usepackage{setspace}
%					\doublespacing

% Zeichensatz
\usepackage[latin1]{inputenc}

% deutsches Woerterbuch
\usepackage[ngerman]{babel}
\usepackage[all]{nowidow}		%Hurenkinder+Schusterjungen unterdr�cken

% Anfuehrungszeichen mit \enquote{}
\usepackage[babel,german=quotes]{csquotes}

% Zeilenabstand
\usepackage{setspace}
\onehalfspacing

% Kopf- und Fusszeile
% Format der Kopf- und Fusszeile
\usepackage{scrpage2}%scrpage2 is obsolete
%\usepackage{scrlayer-scrpage}
\pagestyle{scrheadings}
\setlength{\headheight}{1.1\baselineskip}
%\setheadtopline{1.5pt}
\setheadsepline{current}
\setfootsepline{current}
%\setfootbotline{1.5pt}
\automark{section}
\lehead{\headmark}
\cehead{}
\rehead{\pagemark}
\lefoot{}
\cefoot{}
\refoot{}
\lohead{\headmark}
\cohead{}
\rohead{\pagemark}
\lofoot{}
\cofoot{}
\rofoot{}

% Abstaende zwischen Fussnoten auf 6pt setzen
\renewcommand{\footnotesep}{6pt}

% Grafiken
%\usepackage[final]{graphicx}
%\usepackage{graphicx}
%\makeatletter
%\def\ScaleIfNeeded{%
%	\ifdim\Gin@nat@width>\linewidth
%		\linewidth
%	\else
%		\Gin@nat@width
%	\fi
%}
%\makeatother

% Farben
\usepackage{color}
\definecolor{LinkColor}{rgb}{0.0,0.0,0.0}
\definecolor{HellGruen}{rgb}{.4,.8,.4}
\definecolor{Pink}{rgb}{.77,0,.38}
\definecolor{Lila}{rgb}{.49,0,.49}
\definecolor{Orange}{rgb}{.94,.47,0}


\usepackage[hyphens]{url}
\usepackage[pdftitle={},
	pdfauthor={Louisa Liesner},
	pdfcreator={},
	pdfsubject={},
	pdfkeywords={}]{hyperref}
\hypersetup{colorlinks=true,
	linkcolor=LinkColor,
	citecolor=LinkColor,
	filecolor=LinkColor,
	menucolor=LinkColor,
	urlcolor=LinkColor}

% Mathematische Symbole
\usepackage{amsmath}
\usepackage{amssymb}

% Ma�einhaeiten-Darstellung verbessern
\usepackage{units}

% Literaturverzeichnis

% Abk�rzungsverzeichnis
%\usepackage{acronym}

%\PassOptionsToPackage{hyphens}{url}
\usepackage{hyperref}
\usepackage[numbers]{natbib}
\usepackage[nohyperlinks, printonlyused, smaller]{acronym}%[printonlyused]


% Einstellungen f�r Tabelle
\usepackage{multirow}
\usepackage{tabularx}	%Zeilenumbruch und automatische Breiteneinstellung in Tabellen
\usepackage{longtable}	%lange Tabellen mit Seitenumbruch
\usepackage{ltablex}

% Indexerstellung
\usepackage{makeidx}
\makeindex

% Grafiken
%\usepackage[final]{graphicx}
\usepackage{graphics}
\usepackage{subfigure}
%\usepackage{float}

\makeatletter
\def\ScaleIfNeeded{%
	\ifdim\Gin@nat@width>\linewidth
		\linewidth
	\else
		\Gin@nat@width
	\fi
}
\makeatother

\usepackage{pdfpages}
\usepackage{pdflscape}

\usepackage{microtype} % verhindert, dass �ber den Rand geschrieben wird 
