\section{Infrastrukturelle Umfelderfassung}
\label{sec:TechnikStand}
Dieses Kapitel stellt zun�chst die bestehenden Sensoren zur Umfelderfassung vor und erl�utert deren Funktionsprinzipien. Anschlie�end wird auf die Algorithmen zur Datenfusion, Objekterkennung und Tracking eingegangen. Auf diesen Grundlagen aufbauend wird in den Kapiteln \ref{sec:KFZSensor} und \ref{sec:InfraSensor} aufgezeigt, wie dies im Fahrzeug und auch in der Infrastruktur genutzt wird.

\subsection{Sensoren zur Umfelderfassung}
\label{sec:Sensoren}
Die Sensoren zur Umfelderfassung werden in entfernungsgebende und bildgebende Sensoren unterschieden. Zu ersterem geh�ren der Ultraschall, der Radar und der Lidar. Zu letzterem die Kamera mit dem sichtbaren und dem Infrarotspektrum. Im Folgendem werden die einzelnen Sensorprinzipien erl�utert.

\subsubsection{Ultraschall}
\label{sec:Ultraschall}
Als Ultraschall werden die Schallfrequenzen ab \unit{20}{kHz} bezeichnet. Sie geh�ren zu den Frequenzen, die f�r das menschliche Ohr nicht h�rbar sind. Die Messung mit Ultraschall geh�rt zu den Laufzeitmessungen. Ein Sender emittiert Schallwellen, die schlie�lich von Objekten reflektiert werden. Mit Hilfe der gemessenen Laufzeit $\Delta t$ bis das Echo wieder am Sender ankommt kann der Abstand $d$ zum gemessenen Objekt bestimmt werden:

\begin{equation}
d = \dfrac{c \Delta t}{2}.
\label{eq:Abstand}
\end{equation}

Hierbei ist $c_S$ die Schallgeschwindigkeit. Da die Strecke zwischen Sender und Objekt zweimal durchlaufen wird, muss diese halbiert werden um den tats�chlichen Abstand zu erhalten.

Die Reichweite des Ultraschallsensors ist abh�ngig von der ausgesendeten Schallintensit�t $I_S$, da die Schallintensit�t zum Einen in Abh�ngigkeit von der Entfernung $r$ und zum Anderen durch den Reflexionsgrad $\rho$ des gemessenen Objektes abnimmt. Somit ergibt sich, mit der effektiven Reflexionsfl�che $\sigma$ und bezogen auf den Normabstand $r_1$, die reflektierte Schallintensit�t \marginpar{Schallreflexionsgrad $\rho$?}

\begin{equation}
I_{refl}=\sigma I_s \left(\dfrac{r_1}{2r}\right)^2.
\label{eq:Irefl}
\end{equation}

Damit ein Objekt erkannt wird muss das Empfangssignal oberhalb des Rauschens liegen, d.h. \unit{$\leq$10}{dB} sein.

\subsubsection{Radar}
\label{sec:Radar}
Radar steht f�r \textbf{ra}dio \textbf{d}etection \textbf{a}nd \textbf{r}anging und nutzt die elektromagnetischen Wellen im Radiofrequenzbereich. F�r den Automobilbereich sind die \unit{24}{GHz} und \unit{77}{GHz} Frequenzb�nder reserviert. \marginpar{Quellen!}

Im Gegensatz zum Ultraschall breiten sich hier die Wellen nicht in alle Raumrichtungen gleichm��ig aus sondern werden mit Hilfe einer sogenannten Richtantenne geb�ndelt. Je nach Richtcharakteristik ergibt sich der Antennengewinn $G$, der Einfluss auf die Reichweite nimmt. Die Empfangsleistung f�r ein reflektiertes Radarsignal ergibt sich zu

\begin{equation}
P_R = 10^{-2kr/1000} \cdot \sigma \cdot \lambda^2 \cdot G^2 \cdot V_{mp}^2 \cdot P_{total}/(4\pi)^3 r^4
\label{eq:Empfangsleistung}
\end{equation}

mit dem R�ckstreuquerschnitt

\begin{equation}
\sigma_{plate} = 4\pi \dfrac{A^2}{\lambda^2}.
\label{eq:Rueckstreuquerschnitt}
\end{equation}

Gleichung \ref{eq:Empfangsleistung} ber�cksichtigt au�erdem sogenannte Signalleistungssch�ttler mit dem Faktor $V_{mp}^2$, $0 \leq V_{mp} \leq 2$.

Bei der Abstandsmessung wird neben der Laufzeitbestimmung der Doppler-Effekt genutzt. Der Doppler-Effekt besagt, dass sich die Frequenz bei der Reflexion in Abh�ngigkeit von der �nderung des Abstandes $\dot{r}$ �ndert. Diese Frequenz wird auch Dopplerfrequenz $f_{Doppler}$ genannt und ergibt sich mit der Tr�gerfrequenz $f_0$ und der Lichtgeschwindigkeit $c$ folgenderma�en:

\begin{equation}
f_{Doppler} = - 2 \dot{r} f_0/c
\label{eq:Doppler}
\end{equation}

Bei einer Ann�herung ($\dot{r}<0$) ist diese positiv und beim Entfernen negativ. Stehende Objekte k�nnen mit diesem Effekt jedoch nicht gemessen werden.

\subsubsection{Lidar}
\label{sec:Lidar}
Das Lidar (light detection and ranging) geh�rt zu den optischen Messverfahren und nutzt Laserpulse. Der Abstand wird, wie beim Ultraschall, mittels Laufzeitmessung bestimmt. Die empfangene Lichtintensit�t ist insbesondere von der Gr��e und vom Reflexionsgrad $\rho$ des gemessenen Objektes abh�ngig. So bestimmt sie sich f�r ein gro�es Objekt mit 

\begin{equation}
P_r = \dfrac{\rho \cdot A_t \cdot H \cdot T^2 \cdot P_t}{\pi^2 \cdot R^3 \cdot (Q_v/4)(\Phi/2)^2}
	\label{eq:Pgross}
\end{equation}

und f�r ein kleines Objekt mit

\begin{equation}
P_r = \dfrac{\rho \cdot A_t \cdot H \cdot T^2 \cdot P_t}{\pi^2 \cdot R^4 \cdot (Q_v Q_h/4)(\Phi/2)^2}. 
\label{eq:Pklein}
\end{equation}

\subsubsection{Kamera}
\label{sec:Kamera}

\subsection{Algorithmen zur Datenverarbeitung}
\label{sec:Algorithmen}

\subsubsection{Datenfusion}
\label{sec:Fusion}

\subsubsection{Objekterkennung}
\label{sec:Objekterkennung}

\subsubsection{Tracking}
\label{sec:Tracking}

\subsection{Sensoreinsatz im Fahrzeug}
\label{sec:KFZSensor}

\subsection{Sensoreinsatz in der Infrastruktur}
\label{sec:InfraSensor}