\section{Zusammenfassung und Ausblick}
\label{chap:Zusammenfassung}
Ziel dieser Arbeit ist die Erstellung eines Simulations-Tools zur Auslegung und Bewertung von Sensorsystemkonzpeten zur infrastrukturellen Umfelderfassung im Hinblick auf das automatisierte Fahren. Dazu wurden in den Grundlagen zun�chst die Eigenschaften der Umfelderfassungssensoren Ultraschall, Radar, Lidar und Kamera erarbeitet. Dem schloss sich eine Recherche bez�glich bestehender Sensorkonfigurationen von Fahrzeugen und in der Infrastruktur an. Anhand der erlangten Erkenntnisse wurden Anforderungen an das Simulations-Tool ermittelt. Die Anforderungen wurden schlie�lich in Matlab umgesetzt. Das Ergebnis ist ein Programm mit einer Benutzeroberfl�che, die es dem Benutzer erm�glicht Sensorkonfigurationen auszuwerten. Es k�nnen drei verschiedene Stra�enf�hrungen gew�hlt sowie Objekte und Sensoren platziert werden. F�r die Bewertung der Konfigurationen kann die Tageszeit ausgew�hlt, die Witterung eingestellt, die Verarbeitungslatenz der Infrastruktur und der Objekte angegeben und die Erkennungwahrscheinlichkeit der Objekterkennungsalgorithmen angegeben werden. Anschlie�end werden die erfassten Objekte mit Erkennungswahrscheinlichkeit, Genauigkeit und Latenz ausgegeben. 

Die Anwendbarkeit wurde schlie�lich mit vier verschiedenen Szenen evaluiert. Das Ergebnis dieser Studie zeigte, dass die Einfl�sse Tageszeit, Witterung und Objekterkennungsalgorithmus auf die Objekterkennung ber�cksichtigt werden. Dadurch sind eine erste Auswertung und ein Vergleich von Sensorkonfigurationen bereits m�glich.

Zur Verbesserung der Auswertung der Konfigurationen kann das Simulations-Tool jedoch noch um Funktionen erweitert werden. Weitere Arbeiten k�nnten sich damit besch�ftigen, die verschiedenen Ans�tze der Sensordatenfusion zu implementieren. Im Rahmen dessen k�nnte auch die \acs{C2X}-Kommunikation integriert werden. Au�erdem k�nnte das Tool um eine zeitliche Auswertung erweitert werden, sodass auch der Einfluss des Objekttrackings ber�cksichtigt werden kann. Dadurch k�nnte schlie�lich ausgewertet werden, wann welcher Sensor welches Objekt erfasst. Des Weiteren kann die Objekterkennung noch erweitert werden, indem sichtverdeckende Elemente, wie H�user, eingebaut werden und eine funktionale Abh�ngigkeit f�r Lichtverh�ltnisse ber�cksichtigt wird. Eine Option zum Ein- und Ausblenden der Sensorsichtfelder w�rde zu einer �bersichtlicheren Darstellung der Szene verhelfen.
